
{\actuality} В последнее время всё более очевидной становится тенденция к повышению точности и чувствительности космических средств, предназначенных для наблюдения и получения информации о положении малоэнергетических целей. Одним из путей решения этой задачи является увеличение размеров оптических систем космического назначения.

Расширение эксплуатационных возможностей такой широкоформатной оптики предполагает, в свою очередь, введение в оптическую систему элементов, позволяющих изменять в пространстве положение визирной оси оптической аппаратуры. Эту задачу можно решить либо поворотом космического аппарата (КА) в пространстве, либо за счет изменения положения одного либо нескольких элементов оптической системы относительно КА. Другим вариантом получения эффекта перенацеливания оптической системы является разворот всей оптической системы относительно КА.
~\autocite{Gosele1999161,Lermontov}

Понятно, что с точки зрения экономии энергии на борту КА наиболее рациональным решением будет перемещение в процессе перенацеливания минимальной массы (одного или нескольких элементов оптической системы). Но, независимо от принятого выбора конструктивного исполнения, при изменении положения перемещаемой массы относительно КА возникнут реактивные силы и моменты, воздействующие на КА, которые приведут к развороту КА вокруг его центра масс в направлении, противоположном направлению перемещения подвижной массы. Таким образом, в результате взаимного перемещения оптической системы (или её элементов) относительно КА на некоторый заданный угол и перемещения самого КА в пространстве ось визирования оптической системы займёт в пространстве некоторое положение, не совпадающее с заданными углами на перенацеливание.  Особенно сильно влияние реактивных моментов и сил в случае инфракрасных оптических систем космического назначения, имеющих значительные габариты массу. 

Как правило, на борту КА функционирует система стабилизации положения КА в пространстве. В результате работы этой системы через некоторое время (значительно большее времени перенацеливания) КА вернётся в положение, которое он занимал до начала процесса перенацеливания. Только тогда ось визирования оптической системы постепенно займёт требуемое положение. Кроме того, на КА закреплено большое количество устройств (антенны, солнечные панели и т.д.), имеющих достаточно низкую частоту собственных колебаний. Если перенацеливание происходит за короткое время, то реактивное воздействие на КА близко по своей природе к ударному воздействию, что может привести к возникновению медленно затухающих собственных колебаний этих устройств относительно КА. Это обстоятельство приводит к возникновению дополнительных гармонических воздействий на КА, что затрудняет работу системы стабилизации КА и приводит к ещё большему затягиванию процесса перенацеливания оси визирования оптической системы.
Следует отметить, что данная тематика в настоящее время разработана недостаточно подробно и имеющийся список научной литературы по данному вопросу весьма скуден.

В связи с этим, актуально проведение подробного исследования результатов влияния реактивных моментов на КА, возникающих в процессе перенацеливания визирной оси оптической системы, входящей в состав КА.


\ifsynopsis
Этот абзац появляется только в~автореферате.
Для формирования блоков, которые будут обрабатываться только в~автореферате,
заведена проверка условия \verb!\!\verb!ifsynopsis!.
Значение условия задаётся в~основном файле документа (\verb!synopsis.tex! для
автореферата).
\else
Этот абзац появляется только в~диссертации.
Через проверку условия \verb!\!\verb!ifsynopsis!, задаваемого в~основном файле
документа (\verb!dissertation.tex! для диссертации), можно сделать новую
команду, обеспечивающую появление цитаты в~диссертации, но~не~в~автореферате.
\fi

% {\progress}
% Этот раздел должен быть отдельным структурным элементом по
% ГОСТ, но он, как правило, включается в описание актуальности
% темы. Нужен он отдельным структурынм элемементом или нет ---
% смотрите другие диссертации вашего совета, скорее всего не нужен.

{\aim} данной является разработка и исследование метода расчёта реактивных моментов, возникающих при перенацеливании оси визирования оптических систем космического назначения, разработка и исследование оборудования для измерения реактивных моментов в наземных условиях, проведение расчётов и экспериментальных исследований конкретных образцов оптических систем космического назначения.

Для~достижения поставленной цели необходимо было решить следующие {\tasks}:
\begin{enumerate}[beginpenalty=10000] % https://tex.stackexchange.com/a/476052/104425
  \item Провести анализ и классификацию существующих методов компенсации реактивных воздействий на КА, возникающих при перенацеливании оси визирования оптической системы.
  \item Разработать методику расчёта средств на борту КА, предназначенных для компенсации реактивных моментов, возникающих при перенацеливании оси визирования для конкретных образцов оптических систем космического назначения.
  \item Разработать испытательный стенд для измерения реактивных воздействий на КА при проведении наземных испытаний оптических систем.
  \item Разработать методику проведения наземных испытаний оптических систем космического назначения с точки зрения измерения реактивных воздействий на КА.
\end{enumerate}


{\novelty}
\begin{enumerate}[beginpenalty=10000] % https://tex.stackexchange.com/a/476052/104425
  \item Впервые проведены расчёты пространственных реактивных воздействий на КА при проектировании нескольких вариантов крупногабаритных оптических систем космического назначения с возможностью перенацеливания визирной оси, которые позволили оптимизировать конструкцию аппаратуры и алгоритм управления устройством перенацеливания.
  \item Разработанная испытательная аппаратура и методика проведения испытаний крупногабаритных оптических систем космического назначения с возможностью перенацеливания визирной оси позволила получить значения допустимых реактивных воздействий, требуемые в техническом задании на проектирование, что подтверждено результатами наземных и лётных испытаний. 
\end{enumerate}

{\influence} \ldots

{\methods} \ldots

{\defpositions}
\begin{enumerate}[beginpenalty=10000] % https://tex.stackexchange.com/a/476052/104425
  \item Первое положение
  \item Второе положение
  \item Третье положение
  \item Четвертое положение
\end{enumerate}
В папке Documents можно ознакомиться с решением совета из Томского~ГУ
(в~файле \verb+Def_positions.pdf+), где обоснованно даются рекомендации
по~формулировкам защищаемых положений.

{\reliability} полученных результатов обеспечивается \ldots \ Результаты находятся в соответствии с результатами, полученными другими авторами.


{\probation}
Основные результаты работы докладывались~на:
перечисление основных конференций, симпозиумов и~т.\:п.

{\contribution} Автор принимал активное участие \ldots

\ifnumequal{\value{bibliosel}}{0}
{%%% Встроенная реализация с загрузкой файла через движок bibtex8. (При желании, внутри можно использовать обычные ссылки, наподобие `\cite{vakbib1,vakbib2}`).
    {\publications} Основные результаты по теме диссертации изложены
    в~XX~печатных изданиях,
    X из которых изданы в журналах, рекомендованных ВАК,
    X "--- в тезисах докладов.
}%
{%%% Реализация пакетом biblatex через движок biber
    \begin{refsection}[bl-author, bl-registered]
        % Это refsection=1.
        % Процитированные здесь работы:
        %  * подсчитываются, для автоматического составления фразы "Основные результаты ..."
        %  * попадают в авторскую библиографию, при usefootcite==0 и стиле `\insertbiblioauthor` или `\insertbiblioauthorgrouped`
        %  * нумеруются там в зависимости от порядка команд `\printbibliography` в этом разделе.
        %  * при использовании `\insertbiblioauthorgrouped`, порядок команд `\printbibliography` в нём должен быть тем же (см. biblio/biblatex.tex)
        %
        % Невидимый библиографический список для подсчёта количества публикаций:
        \phantom{\printbibliography[heading=nobibheading, section=1, env=countauthorvak,          keyword=biblioauthorvak]%
        \printbibliography[heading=nobibheading, section=1, env=countauthorwos,          keyword=biblioauthorwos]%
        \printbibliography[heading=nobibheading, section=1, env=countauthorscopus,       keyword=biblioauthorscopus]%
        \printbibliography[heading=nobibheading, section=1, env=countauthorconf,         keyword=biblioauthorconf]%
        \printbibliography[heading=nobibheading, section=1, env=countauthorother,        keyword=biblioauthorother]%
        \printbibliography[heading=nobibheading, section=1, env=countregistered,         keyword=biblioregistered]%
        \printbibliography[heading=nobibheading, section=1, env=countauthorpatent,       keyword=biblioauthorpatent]%
        \printbibliography[heading=nobibheading, section=1, env=countauthorprogram,      keyword=biblioauthorprogram]%
        \printbibliography[heading=nobibheading, section=1, env=countauthor,             keyword=biblioauthor]%
        \printbibliography[heading=nobibheading, section=1, env=countauthorvakscopuswos, filter=vakscopuswos]%
        \printbibliography[heading=nobibheading, section=1, env=countauthorscopuswos,    filter=scopuswos]}%
        %
        \nocite{*}%
        %
        {\publications} Основные результаты по теме диссертации изложены в~\arabic{citeauthor}~печатных изданиях,
        \arabic{citeauthorvak} из которых изданы в журналах, рекомендованных ВАК%
        \ifnum \value{citeauthorscopuswos}>0%
            , \arabic{citeauthorscopuswos} "--- в~периодических научных журналах, индексируемых Web of~Science и Scopus%
        \fi%
        \ifnum \value{citeauthorconf}>0%
            , \arabic{citeauthorconf} "--- в~тезисах докладов.
        \else%
            .
        \fi%
        \ifnum \value{citeregistered}=1%
            \ifnum \value{citeauthorpatent}=1%
                Зарегистрирован \arabic{citeauthorpatent} патент.
            \fi%
            \ifnum \value{citeauthorprogram}=1%
                Зарегистрирована \arabic{citeauthorprogram} программа для ЭВМ.
            \fi%
        \fi%
        \ifnum \value{citeregistered}>1%
            Зарегистрированы\ %
            \ifnum \value{citeauthorpatent}>0%
            \formbytotal{citeauthorpatent}{патент}{}{а}{}%
            \ifnum \value{citeauthorprogram}=0 . \else \ и~\fi%
            \fi%
            \ifnum \value{citeauthorprogram}>0%
            \formbytotal{citeauthorprogram}{программ}{а}{ы}{} для ЭВМ.
            \fi%
        \fi%
        % К публикациям, в которых излагаются основные научные результаты диссертации на соискание учёной
        % степени, в рецензируемых изданиях приравниваются патенты на изобретения, патенты (свидетельства) на
        % полезную модель, патенты на промышленный образец, патенты на селекционные достижения, свидетельства
        % на программу для электронных вычислительных машин, базу данных, топологию интегральных микросхем,
        % зарегистрированные в установленном порядке.(в ред. Постановления Правительства РФ от 21.04.2016 N 335)
    \end{refsection}%
    \begin{refsection}[bl-author, bl-registered]
        % Это refsection=2.
        % Процитированные здесь работы:
        %  * попадают в авторскую библиографию, при usefootcite==0 и стиле `\insertbiblioauthorimportant`.
        %  * ни на что не влияют в противном случае
        \nocite{vakbib2}%vak
        \nocite{patbib1}%patent
        \nocite{progbib1}%program
        \nocite{bib1}%other
        \nocite{confbib1}%conf
    \end{refsection}%
        %
        % Всё, что вне этих двух refsection, это refsection=0,
        %  * для диссертации - это нормальные ссылки, попадающие в обычную библиографию
        %  * для автореферата:
        %     * при usefootcite==0, ссылка корректно сработает только для источника из `external.bib`. Для своих работ --- напечатает "[0]" (и даже Warning не вылезет).
        %     * при usefootcite==1, ссылка сработает нормально. В авторской библиографии будут только процитированные в refsection=0 работы.
}

При использовании пакета \verb!biblatex! будут подсчитаны все работы, добавленные
в файл \verb!biblio/author.bib!. Для правильного подсчёта работ в~различных
системах цитирования требуется использовать поля:
\begin{itemize}
        \item \texttt{authorvak} если публикация индексирована ВАК,
        \item \texttt{authorscopus} если публикация индексирована Scopus,
        \item \texttt{authorwos} если публикация индексирована Web of Science,
        \item \texttt{authorconf} для докладов конференций,
        \item \texttt{authorpatent} для патентов,
        \item \texttt{authorprogram} для зарегистрированных программ для ЭВМ,
        \item \texttt{authorother} для других публикаций.
\end{itemize}
Для подсчёта используются счётчики:
\begin{itemize}
        \item \texttt{citeauthorvak} для работ, индексируемых ВАК,
        \item \texttt{citeauthorscopus} для работ, индексируемых Scopus,
        \item \texttt{citeauthorwos} для работ, индексируемых Web of Science,
        \item \texttt{citeauthorvakscopuswos} для работ, индексируемых одной из трёх баз,
        \item \texttt{citeauthorscopuswos} для работ, индексируемых Scopus или Web of~Science,
        \item \texttt{citeauthorconf} для докладов на конференциях,
        \item \texttt{citeauthorother} для остальных работ,
        \item \texttt{citeauthorpatent} для патентов,
        \item \texttt{citeauthorprogram} для зарегистрированных программ для ЭВМ,
        \item \texttt{citeauthor} для суммарного количества работ.
\end{itemize}
% Счётчик \texttt{citeexternal} используется для подсчёта процитированных публикаций;
% \texttt{citeregistered} "--- для подсчёта суммарного количества патентов и программ для ЭВМ.

Для добавления в список публикаций автора работ, которые не были процитированы в
автореферате, требуется их~перечислить с использованием команды \verb!\nocite! в
\verb!Synopsis/content.tex!.
